
\documentclass[12pt]{report}

%%%%%%%%%%%%%%%%%%%%%%%%%%%%%%%%%%%%%%%%%%%%%%%%%%%%%%%%%%%%%%%%%%%%
% package and document formatting stuff
%%%%%%%%%%%%%%%%%%%%%%%%%%%%%%%%%%%%%%%%%%%%%%%%%%%%%%%%%%%%%%%%%%%%

\usepackage{amsmath,amsthm,amsfonts,amscd,amssymb,mathabx}
\usepackage{makeidx,enumerate}

\usepackage{fullpage,color}
\usepackage[pdfstartview=FitH,
%             pdfauthor={\myauthor},
%             pdftitle={\mytitle},
            colorlinks,
            linkcolor=reference,
            citecolor=citation,
            urlcolor=e-mail,
            backref]{hyperref}
\usepackage[all]{xy}

\definecolor{todo}{rgb}{.80,.20,.20}
\definecolor{e-mail}{rgb}{0,.40,.80}
\definecolor{reference}{rgb}{.10,.40,.42}
\definecolor{mrnumber}{rgb}{.80,.40,0}
\definecolor{citation}{rgb}{0,.40,.80}

%%%%%%%%%%%%%%%%%%%%%%%%%%%%%%%%%%%%%%%%%%%%%%%%%%%%%%%%%%%%%%%%%%%%
% theorem stuff
%%%%%%%%%%%%%%%%%%%%%%%%%%%%%%%%%%%%%%%%%%%%%%%%%%%%%%%%%%%%%%%%%%%%

\theoremstyle{plain}

\newtheorem{thm}{Theorem}[section]
\newtheorem{defn}[thm]{Definition}
\newtheorem{deflem}[thm]{Definition/Lemma}
\newtheorem{notn}[thm]{Notation}
\newtheorem{lem}[thm]{Lemma}
\newtheorem{aside}[thm]{Aside}
\newtheorem{rem}[thm]{Remark}
\newtheorem{ex}[thm]{Example}
\newtheorem{facts}[thm]{Facts}
\newtheorem{cor}[thm]{Corollary}
\newtheorem{conj}[thm]{Conjecture}
\newtheorem{prop}[thm]{Proposition}


%%%%%%%%%%%%%%%%%%%%%%%%%%%%%%%%%%%%%%%%%%%%%%%%%%%%%%%%%%%%%%%%%%%%
% typography stuff
%%%%%%%%%%%%%%%%%%%%%%%%%%%%%%%%%%%%%%%%%%%%%%%%%%%%%%%%%%%%%%%%%%%%

\newcommand{\mb}[1]{\mathbf #1}
\newcommand{\mbb}[1]{\mathbb #1}
\newcommand{\mf}[1]{\mathfrak #1}
\newcommand{\mc}[1]{\mathcal #1}
\newcommand{\ms}[1]{\mathscr #1}
\newcommand{\mcu}[1]{\mathcu #1}
\newcommand{\oper}[1]{\operatorname{#1}}

\newcommand{\da}{\downarrow}
\newcommand{\ra}{\rightarrow}
\newcommand{\hra}{\hookrightarrow}
\newcommand{\dra}{\dashrightarrow}
\newcommand{\la}{\leftarrow}
\newcommand{\lra}{\longrightarrow}

\newcommand{\ov}{\overline}
\newcommand{\til}{\widetilde}
\newcommand{\wh}{\widehat}

\newcommand{\ZZ}{\mathbb{Z}}

\newcommand{\ann}{\oper{ann}}
\newcommand{\coker}{\oper{coker}}
\newcommand{\End}{\oper{End}}
\newcommand{\Aut}{\oper{Aut}}
\newcommand{\Stab}{\oper{Stab}}

\newcommand{\ind}{\oper{ind}}
\newcommand{\per}{\oper{per}}
\newcommand{\cores}{\oper{cor}}

\newcommand{\Br}{\oper{Br}}
\newcommand{\quat}[3]{
  \left(\begin{matrix} #1, #2 \\ #3
  \end{matrix}\right)
}
\newcommand{\symb}[3]{
  \left(#1, #2\right)_{#3}
}

\newcommand{\lcm}{\oper{lcm}}

%%%%%%%%%%%%%%%%%%%%%%%%%%%%%%%%%%%%%%%%%%%%%%%%%%%%%%%%%%%%%%%%%%%%
% other stuff
%%%%%%%%%%%%%%%%%%%%%%%%%%%%%%%%%%%%%%%%%%%%%%%%%%%%%%%%%%%%%%%%%%%%

\makeindex
\newcommand{\X}[1]{#1\index{#1}}
\newcommand{\Xb}[1]{\textbf{#1}\index{#1}}

\newcommand{\todo}[1]{\textcolor{todo}{#1}}

%%%%%%%%%%%%%%%%%%%%%%%%%%%%%%%%%%%%%%%%%%%%%%%%%%%%%%%%%%%%%%%%%%%%
% end preamble 
%%%%%%%%%%%%%%%%%%%%%%%%%%%%%%%%%%%%%%%%%%%%%%%%%%%%%%%%%%%%%%%%%%%%

\begin{document}

%%%%%%%%%%%%%%%%%%%%%%%%%%%%%%%%%%%%%%%%%%%%%%%%%%%%%%%%%%%%%%%%%%%%
% title stuff
%%%%%%%%%%%%%%%%%%%%%%%%%%%%%%%%%%%%%%%%%%%%%%%%%%%%%%%%%%%%%%%%%%%%


\author{Daniel Krashen}
\title{Field Arithemtic Notes}

\maketitle
\tableofcontents

%%%%%%%%%%%%%%%%%%%%%%%%%%%%%%%%%%%%%%%%%%%%%%%%%%%%%%%%%%%%%%%%%%%%
% document stuff
%%%%%%%%%%%%%%%%%%%%%%%%%%%%%%%%%%%%%%%%%%%%%%%%%%%%%%%%%%%%%%%%%%%%

\chapter{Philosophy}

Fields are important due to their ubiquity. While they arise in a number of
different contexts in different branches of mathematics, field theory gives
us a coherent set of tools in which to view structural properties.

It is fair to say that field theory is very much a work in progress. While
the arithmetic of the rational numbers and number fields more generally, is
a rich and exciting field, using a broad spectrum of techniques from
algebra, analysis, and even topology, the study of more general fields
raises many questions of a more foundational nature, as we try to
generalize many strategies and concepts which one can take for granted in
the case of number fields. The primary source for fields are
\begin{enumerate}
\item function fields of varieties
\item finitely generated fields, coming from number theory
\item fields of meromorphic functions on analytic manifolds
\end{enumerate}
In addition, a rich source of (slightly more technical) examples, are
fields constructed by taking inductive limits of other fields. Such
constructions play an important role in producing counterexamples, and in
intermediate steps in arguments -- often, and somewhat counterintuitively,
fields may become structurally simpler as they get larger (consider
$\mathbb C$ versus $\mathbb Q$).

For various classes of fields, some examples of fundamental questions which
would like to understand are as follows (followed by some names of formal
ideas we will use to address them):
\begin{itemize}
\item what geometric/topological notions of ``size'' or ``closeness'' make
sense? 

(\textbf{valuations and completions})

\item what notions of dimension do we have? 

(\textbf{cohomological
dimension, Diophantine dimension, Brauer dimension, transcendence degrees,
$p$-bases, ...})

\item can we make sense of positivity? How many ways can we make sense of
this notion? 

(\textbf{real orderings and the Harrison topology})

\item what kinds of structural constraints do Galois groups have?

(\textbf{the inverse Galois problem})

\item To what extent can we concretely describe how Galois extensions are
constructed (for example, with a given group $G$)? What about other
specific types of field extensions? 

(\textbf{generic Galois theory})

\item When can we interpret elements of fields as (rational) functions on a
variety/space? 

(\textbf{Grothendieck's Anabelian conjectures})
\end{itemize}

From another perspective, a very important tool in studying a field, is to
consider the behaviour of algebraic objects over that field, such as
algebras, quadratic forms, algebraic varieties, etcetera. The behaviour of
such objects tells us a great deal about what the arithmetic of the field
is like. Philosophically, one can think of the utility of this point of
view in the following way: as mathematicians, a basic starting point is to
ask the question ``given a collection of polynomial equations, with
coefficients in my field $F$, can I find a solution in $F$?'' Which
equations do and which equations don't have solutions represents a crucial
kind of information about the field. As evidence for this, consider that
the foundational notion of a field extension is based on the procedure of
formally adjoining a solution to a single polynomial equation in one
variable. On the other hand, there are a great deal of polynomials
equations in the world, and if we consider arbitrary collections of such
things, it is (in my own opinion) a bit too much for our limited minds to
handle. For this reason, it makes sense to consider systems of equations
for which we can assign strong conceptual meaning to their solutions. A
source for such systems of equations of primary importance turns out to be
algebraic objects: for example we might consider a system of equations for
which a solution represents a zerodivisor in an algebra, an ideal in an
algebra, or a subspace of a vector space on which some quadratic form
vanishes.

A huge benefit of this perspective is the fact that these algebraic
structures come with new rich sets of tools which can therefore be
leveraged to understand the field's arithmetic. That is to say, by
reinterpreting various problems in field arithmetic in terms of algebraic
structures, we are able to leverage our understanding of these algebraic
structures to get new information. We will see various examples of this:
certain noncommutative algebras will give us ways to discuss
relations between cyclic extensions, and quadratic forms will give us a
perspective on a surprisingly wide array of arithmetic questions.

Further, as we will see later, the Galois cohomology groups (which play a
role for a field which is in some sense analogous to the singular
cohomology groups of a topological space) give rise to a rich collection of
both field invariants, as well as invariants of algebraic structures. This
additional connection has been crucial in recent developments in field
theory, most notably in the proof of the Milnor conjecture by Voevodsky, as
well as the establishmenent of the Norm Residue Isomorphism Theorem (the
Bloch-Kato conjecture) by Voevodsky, Rost, Weibel, Suslin, Joukhovitski,
and others.

From a related perspective, instead of looking at systems of equations
whose solutions have these natural interpretation, one might also consider
systems of equations that are at least simple to write down. The most
obvious of these are those given by a single homogeneous polynomial. The
study of when these have nontrivial solutions is the topic of Tsen-Lang
theory, of the so-called $C_i$-property, and is encapsulated in the notion
of ``Diophantine Dimension.''

\bigskip

The object of these notes is to give, after a brief refresher of field
theory, a guide to these topics, with a particular eye towards Galois
cohomology.

\chapter{Basics}

\section{Some useful algebraic structures}

\begin{defn}
Monoid
\end{defn}

\begin{defn}
Cancellative Monoid
\end{defn}

\begin{defn}
Group
\end{defn}

\begin{defn}
Ring
\end{defn}

\begin{defn}
Commutative Integral Domain
\end{defn}

\begin{defn}
Field
\end{defn}

\section{Fields}

\begin{defn}
A prime field is a field which contains no proper subfields.
\end{defn}

\begin{prop}
The only prime fields are $\mathbb F_p = \mathbb Z/p \mathbb Z$ for $p$ a
prime number, and $\mathbb Q$.
\end{prop}
\begin{proof}
Consider the (unique) homomorphism $\mathbb Z \to F$.
\end{proof}

\begin{defn}
The characteristic of a field is nonnegative generator of the kernel of the
map $\mathbb Z \to F$.
\end{defn}

\begin{defn}
If $F$ is a subfield of $L$, we say that $L$ is a field extension of $F$
and write $L/F$.
\end{defn}

\begin{defn}
Simple extension
\end{defn}

\begin{defn}
Splitting field
\end{defn}

\begin{lem}
Dedekind Lemma and corollary
\end{lem}

\begin{defn}
Normality ( = splitting field)
\end{defn}

\begin{defn}
separability (and derivatives)
\end{defn}

\begin{defn}
Galois extensions:
\begin{itemize}
\item normal + separable
\item num auts = degree
\item $(E, G, 1)$ iso to $Hom_F(E, E)$
\end{itemize}
\end{defn}

\begin{thm}
Galois correspondence
\end{thm}

\section{Infinite things}

\begin{defn}
Inductive limits = direct limits $\subset$ colimits
\end{defn}

\begin{defn}
inverse limits = projective limits $\subset$ limits
\end{defn}

Inverse limits come with natural ``topologies'' in many situations.

Direct limits can inheret topologies.


\iffalse

\section{Tensor producs}

\begin{defn}
Let $V$ and $W$ be $F$-vector spaces. The tensor product $V \otimes W = V
\otimes_F W$ is the vector space generated by all symbols of the form $v
\otimes w$, subject to the relations
\begin{enumerate}
\item $\lambda (v \otimes w) = \lambda v \otimes w = v \otimes \lambda w$
for $\lambda \in F$, $v, w \in V$
\item $(v + v') \otimes w = v \otimes w + v' \otimes w$ for $v, v', w \in
V$
\item $v \otimes (w + w') = v \otimes w + v \otimes w'$ for $v, w, w' \in
V$
\end{enumerate}
\end{defn}

\begin{prop}
Suppose that $V$ and $W$ are vector spaces with bases $\{v_i\}_{i \in I}$
and $\{w_j\}_{j \in J}$ respectively. Then $V \otimes_F W$ has a basis given
by $\{v_i \otimes w_j\}_{(i, j) \in I \times J}$. In particular, if $V$ and
$W$ are finite dimensional, then $\dim_F(V \otimes_F W) = (\dim_F V)(\dim_F
W)$.
\end{prop}
\begin{proof}
Consider the maps $f_k : V \otimes W \to W$ and $g_\ell: V \otimes W \to V$
given by 
\[f_k(\sum_{i,j} v_i \otimes w_j) = \sum  \]
\end{proof}

\begin{defn}
Let $F$ be a field. An (associative, unital) $F$-algebra is a $F$-vector
space $A$, together with a multiplication $A$
\end{defn}


\fi

\iffalse

\subsection{Modules and bimodules}

\begin{defn}
Let $R$ be a ring. A left $R$-module is a set $M$ together with a binary
operation
\begin{align*}
R \times M &\to M \\
(r, m) &\to rm
\end{align*}
such that
\begin{enumerate}[1. ]
\item $1 m = m$, 
\item $(r_1 r_2) m = r_1 (r_2 m)$, 
\item $(r_1 + r_2)m = r_1m + r_2 m$
\item $r(m_1 + m_2) = rm_1 + rm_2$
\end{enumerate}
\end{defn}

\begin{defn}
Let $R$ be a ring. A right $R$-module is a set $M$ together with a binary
operation
\begin{align*}
M \times R &\to M \\
(m, r) &\to mr
\end{align*}
such that
\begin{enumerate}[1. ]
\item $m 1 = m$, 
\item $m (r_1 r_2) = (m r_1) r2$, 
\item $m (r_1 + r_2)m = m r_1 + m r_2$
\item $(m_1 + m_2)r = m_1 r + m_2 r$
\end{enumerate}
\end{defn}

\begin{notn}
We will occasionally write $M_R$ (respectively $_R M$) to denote the fact
that $M$ is a right (respectively left) $R$-module.
\end{notn}

\begin{rem}
Recall that for a ring $R$, we may define its opposite $R^{op}$ as the ring
with the same underlying set and addition, but with the new multiplication
rule $\cdot$ defined by $r \cdot s = sr$. In this way, we see
that if $M$ is a left $R$ module, then we may define the structure of a
right $R^{op}$ module on $M$ via $m \cdot r = rm$. This gives an
equivalence of categories between left (right) $R$-modules and right (left)
$R^{op}$ modules.
\end{rem}

\begin{defn}
Let $R, S$ be rings. An $R-S$ \X{bimodule} is a set $M$ endowed with a left
$R$-module structure and a right $S$-module structure such that for all $r
\in R, s \in S, m \in M$, we have
\[ r(m s) = (r m) s.\]
\end{defn}
\fi

%\bibliographystyle{alpha}
%\bibliography{citations}
\printindex

\end{document}
